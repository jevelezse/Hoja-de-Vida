%%%%%%%%%%%%%%%%%
% This is an example CV created using altacv.cls (v1.1.5, 1 December 2018) written by
% LianTze Lim (liantze@gmail.com), based on the
% Cv created by BusinessInsider at http://www.businessinsider.my/a-sample-resume-for-marissa-mayer-2016-7/?r=US&IR=T
%
%% It may be distributed and/or modified under the
%% conditions of the LaTeX Project Public License, either version 1.3
%% of this license or (at your option) any later version.
%% The latest version of this license is in
%%    http://www.latex-project.org/lppl.txt
%% and version 1.3 or later is part of all distributions of LaTeX
%% version 2003/12/01 or later.
%%%%%%%%%%%%%%%%

%% If you are using \orcid or academicons
%% icons, make sure you have the academicons
%% option here, and compile with XeLaTeX
%% or LuaLaTeX.
% \documentclass[10pt,a4paper,academicons]{altacv}

%% Use the "normalphoto" option if you want a normal photo instead of cropped to a circle
% \documentclass[10pt,a4paper,normalphoto]{altacv}

\documentclass[10pt,a4paper,ragged2e,]{altacv}

%% AltaCV uses the fontawesome and academicon fonts
%% and packages.
%% See texdoc.net/pkg/fontawecome and http://texdoc.net/pkg/academicons for full list of symbols. You MUST compile with XeLaTeX or LuaLaTeX if you want to use academicons.

% Change the page layout if you need to
\geometry{left=1cm,right=9cm,marginparwidth=6.8cm,marginparsep=1.2cm,top=1.25cm,bottom=1.25cm}

% Change the font if you want to, depending on whether
% you're using pdflatex or xelatex/lualatex
\ifxetexorluatex
  % If using xelatex or lualatex:
  \setmainfont{Carlito}
\else
  % If using pdflatex:
  \usepackage[utf8]{inputenc}
  \usepackage[T1]{fontenc}
  \usepackage[default]{lato}
\fi

% Change the colours if you want to
\definecolor{VividPurple}{HTML}{020202}
\definecolor{SlateGrey}{HTML}{2E2E2E}
\definecolor{LightGrey}{HTML}{666666}
\colorlet{heading}{VividPurple}
\colorlet{accent}{VividPurple}
\colorlet{emphasis}{SlateGrey}
\colorlet{body}{LightGrey}

% Change the bullets for itemize and rating marker
% for \cvskill if you want to
\renewcommand{\itemmarker}{{\small\textbullet}}
\renewcommand{\ratingmarker}{\faCircle}

%% sample.bib contains your publications
\addbibresource{sample.bib}

\begin{document}
\name{Jennifer Vélez Segura}
\tagline{Magister en Bioinformática}
% Cropped to square from https://en.wikipedia.org/wiki/Marissa_Mayer#/media/File:Marissa_Mayer_May_2014_(cropped).jpg, CC-BY 2.0
\photo{2.5cm}{hvfoto1}
\personalinfo{%
  % Not all of these are required!
  % You can add your own with \printinfo{symbol}{detail}
  \email{jennifervs.bio@gmail.com}
  \phone{314-312-7510}
  \mailaddress{Avenida Carrera 30 No 39A -60 Apt. 201}
  \location{Bogotá D.C, Colombia}
  \homepage{https://github.com/jevelezse}
  \linkedin{linkedin.com/in/jennifer-velez-segura-34502958/}
%   \github{github.com/mmayer} % I'm just making this up though.
%   \orcid{orcid.org/0000-0000-0000-0000} % Obviously making this up too. If you want to use this field (and also other academicons symbols), add "academicons" option to \documentclass{altacv}
\\
\cvsection[]{PROFILE}
\justify
{Bióloga con conocimientos en citogenética clásica (cariotipos en médula ósea, muestras de líquido y sangre amniótica) y Máster en Bioinformática. Habilidades en programación con Python, R y Shell para el análisis de datos biológicos y el uso de técnicas de minería de datos y de computación de alto desempeño. Experiencia con análisis de exoma completo y genoma completo para soporte diagnóstico. Interesada en aplicar técnicas de aprendizaje automático en datos biológicos.}
}

%% Make the header extend all the way to the right, if you want.
\begin{fullwidth}
\makecvheader
\end{fullwidth}

%% Depending on your tastes, you may want to make fonts of itemize environments slightly smaller
\AtBeginEnvironment{itemize}{\small}

%% Provide the file name containing the sidebar contents as an optional parameter to \cvsection.
%% You can always just use \marginpar{...} if you do
%% not need to align the top of the contents to any
%% \cvsection title in the "main" bar.

\cvsection[page1sidebar]{Experiencia}

\cvevent{Citogenetista y Bioinformática }{Genetix S.A.S}{Enero 2015 -- Mayo 2020}{Bogotá D.C., Colombia}
\begin{itemize}
\item Análisis citogenético de muestras de sangre periférica, médula ósea, líquido amniótico y vellosidad corial.
\item Diseño e implementación de pipelines bioinformáticos para el análisis de secuencias biológicas.
\item Análisis de exomas y paneles de secuenciación (NGS).
\end{itemize}

\divider

\cvevent{Citogenetista}{Biogenética Diagnostica SAS}{May 2013 -- Nov 2014}{Bogotá D.C., Colombia}
\begin{itemize}
\item Análisis citogenético de muestras de sangre periférica, médula ósea, líquido amniótico y vellosidad corial.
\item Siembra y cosecha de muestras en sangre periférica y médula ósea.
\end{itemize}

\divider

\cvevent{Citogenetista}{Laboratorio de Genética y Biología Molecular}{2013 --  2012}{Bogotá D.C., Colombia}

\begin{itemize}
\item Análisis citogenético de muestras de sangre periférica, médula ósea, líquido amniótico y vellosidad corial.
\item Siembra y cosecha de muestras en sangre periférica y médula ósea.
\item Aislamiento de ADN y preparación de geles en Agarosa.

\end{itemize}

% \divider

% \cvevent{Product Engineer}{Google}{23 June 1999 -- 2001}{Palo Alto, CA}

% \begin{itemize}
% \item Joined the company as employe \#20 and female employee \#1
% \item Developed targeted advertisement in order to use user's search queries and show them related ads
% \end{itemize}

% Adapted from @Jake's answer from http://tex.stackexchange.com/a/82729/226
% \wheelchart{outer radius}{inner radius}{
% comma-separated list of value/text width/color/detail}
% Some ad-hoc tweaking to adjust the labels so that they don't overlap

\cvsection{Filosofía de vida}
\begin{quote}
``Los problemas son parte de la vida y la busqueda de soluciones también."
\end{quote}

\cvsection{Logros}

\cvachievement{\faTrophy}{Aprender a programar}{para poder desarrollar soluciones a problemas biológicos.}

\divider

\cvachievement{\faTrophy}{Aprender Machine Learning}{y poderlo aplicar a la para entender diferentes tipos de datos}

\divider

\cvachievement{\faTrophy}{Ser conferencista en Pycon Colombia}{mostrar a otras personas lo que se puede hacer con Python y R.}
\divider

\cvachievement{\faTrophy}{Ganadora de retos en ciencia de datos:}{El primer reto Stem Sports y Big Data Colombia 2016 y Ganadora de la primera Hackaton scotiabank grupo Segmentación de clientes 2018}
\divider

\cvachievement{\faFemale}{Ser guía de Django Girls Colombia}{Ser invitada como guía ha sido una de las experiencias más gratificante, ya que aprendí y ayude a otras personas.}
\divider

\cvachievement{\faFemale}{Cabana Travel Fellowships}{Beca para presentar el modelo inicial de mi tesis de maestría en Chile. 2018}
\divider

%\clearpage

\cvsection[page2sidebar]{Publicaciones}

\nocite{*}


\printbibliography[heading=pubtype,title={\printinfo{\faFileTextO}{Journal Articles}}, type=article]

\divider

\printbibliography[heading=pubtype,title={\printinfo{\faGroup}{Conference Proceedings}},type=inproceedings]

%% If the NEXT page doesn't start with a \cvsection but you'd
%% still like to add a sidebar, then use this command on THIS
%% page to add it. The optional argument lets you pull up the
%% sidebar a bit so that it looks aligned with the top of the
%% main column.
% \addnextpagesidebar[-1ex]{page3sidebar}


\end{document}
